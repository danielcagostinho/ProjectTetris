\documentclass[12pt, titlepage]{article}
\usepackage{fullpage}
\usepackage[round]{natbib}
\usepackage{multirow}
\usepackage{booktabs}
\usepackage{tabularx}
\usepackage{graphicx}
\usepackage{float}
\usepackage{hyperref}

\hypersetup{
    colorlinks,
    citecolor=black,
    filecolor=black,
    linkcolor=black,
    urlcolor=blue
}
\usepackage[round]{natbib}
\newcounter{acnum}
\newcommand{\actheacnum}{AC\theacnum}
\newcommand{\acref}[1]{AC\ref{#1}}
\newcounter{ucnum}
\newcommand{\uctheucnum}{UC\theucnum}
\newcommand{\uref}[1]{UC\ref{#1}}
\newcounter{mnum}
\newcommand{\mthemnum}{M\themnum}
\newcommand{\mref}[1]{M\ref{#1}}
\title{SE 3XA3: Design Document\\PROJECT TETRIS}
\author{Team \#11, Team Tetris
		\\ Daniel Agostinho agostd
		\\ Anthony Chang changa7
		\\ Divya Sridhar sridhad
}
\date{\today}

\begin{document}
\maketitle
\pagenumbering{roman}
\tableofcontents
\listoftables
\listoffigures
\begin{table}[bp]
\caption{\bf Revision History}
\begin{tabularx}{\textwidth}{p{3cm}p{2cm}p{4cm}X}
\toprule {\bf Date} & {\bf Version} & {\bf Developer(s)} & {\bf Notes}\\
\midrule
11/02/2016 & 0.0 & Anthony Chang & Initial Draft\\
11/02/2016 & 0.0 &  Daniel Agostinho & Initial Draft\\
11/02/2016 & 0.0 & Divya Sridhar & Initial Draft\\
\bottomrule
\end{tabularx}
\end{table}
\newpage
\pagenumbering{arabic}

\section{Introduction}
	This is the Module Guide (MG) for PROJECT TETRIS. PROJECT TETRIS is a reimplementation of the classic arcade game Tetris. The user moves pieces called Tetraminos into a game board called a "well" and stacks them. When a row of the well is completely filled, that row is cleared and the user gets points. The purpose of the game is to obtain a high score by clearing as many rows as possible without letting the stack of pieces reach the top. 
\paragraph{}
	This project was designed with key programming principles in mind. These principles are: information hiding, modularity, and design for change. We divided the project into modules which are independent sections of code that have their own unique purpose. Modules support good programming principles such as information hiding. The modules abstract their information from other modules such that the other modules need not worry about that information. It does not concern them and allows for the modules to be modified independently and changes can be made without affecting the whole system.
\paragraph{}
	The rest of the document is organized as follows. Section
\ref{SecChange} lists the anticipated and unlikely changes of the software
requirements. Section \ref{SecMH} summarizes the module decomposition that
was constructed according to the likely changes. Section \ref{SecConnection}
specifies the connections between the software requirements and the
modules. Section \ref{SecMD} gives a detailed description of the
modules. Section \ref{SecTM} includes two traceability matrices. One checks
the completeness of the design against the requirements provided in the SRS. The
other shows the relation between anticipated changes and the modules. Section
\ref{SecUse} describes the use relation between modules.


\section{Anticipated and Unlikely Changes} \label{SecChange}
This section lists any possible changes to the system. According to the likeliness
of the change, they are classified into two
categories: anticipated changes are listed in Section \ref{SecAchange}, and
unlikely changes are listed in Section \ref{SecUchange}.
\subsection{Anticipated Changes} \label{SecAchange}
A design for change paradigm was utilized during our design process for PROJECT TETRIS.
The following anticipated changes are design choices that are made to elements hidden within 
modules and thus are easy to alter without affecting other components of the project.
\begin{description}
\item[\refstepcounter{acnum} \actheacnum \label{acHardware}:] The specific
  hardware on which the software is running.
\item[\refstepcounter{acnum} \actheacnum \label{acOS}:] The specific
  Operating System on which the software interfaces with.
\item[\refstepcounter{acnum} \actheacnum \label{acJava}:] The language
  of Java (future versions/releases).
\end{description}
\subsection{Unlikely Changes} \label{SecUchange}
The following design choices are incorporated in multiple components of the system. If
a decision should later need to be changed, many different modules will need to be 
modified. Thus, these design decisions are unlikely to change.
\begin{description}
\item[\refstepcounter{ucnum} \uctheucnum \label{ucIO}:] Input/Output devices
  (Input: Mouse and/or Keyboard, Output: File and/or Screen).
\item[\refstepcounter{ucnum} \uctheucnum \label{ucAlgorithm}:] The Algorithm
  for tetromino interaction (setting to well and/or clearing a row).
\item[\refstepcounter{ucnum} \uctheucnum \label{ucGoal}:] Goal of the software:
  Playable Tetris game.
\end{description}


\section{Module Hierarchy} \label{SecMH}
This section provides an overview of the module design. Modules are summarized
in a hierarchy decomposed by secrets in Table \ref{TblMH}. The modules listed
below, which are leaves in the hierarchy tree, are the modules that will
actually be implemented.
\begin{description}
\item [\refstepcounter{mnum} \mthemnum \label{mHH}:] Hardware-Hiding Module
\item [\refstepcounter{mnum} \mthemnum \label{mOM}:] Output Module
\item [\refstepcounter{mnum} \mthemnum \label{mIM}:] Input Module
\item [\refstepcounter{mnum} \mthemnum \label{mGL}:] Game Logic Module
\item [\refstepcounter{mnum} \mthemnum \label{mGG}:] Game Graphics Module
\item [\refstepcounter{mnum} \mthemnum \label{mIC}:] Input Control Module
\end{description}
\begin{table}[h!]
\centering
\begin{tabular}{p{0.3\textwidth} p{0.6\textwidth}}
\toprule
\textbf{Level 1} & \textbf{Level 2}\\
\midrule
{Hardware-Hiding Module} & ~ \\
\midrule
\multirow{4}{0.3\textwidth}{Behaviour-Hiding Module} & Output Module\\
& Input Module\\
& Game Graphics Module\\
& Input Control Module\\
\midrule
\multirow{1}{0.3\textwidth}{Software Decision Module} &  Game Logic\\

\bottomrule
\end{tabular}
\caption{Module Hierarchy}
\label{TblMH}
\end{table}


\section{Connection Between Requirements and Design} \label{SecConnection}
The design of the system is intended to satisfy the requirements developed in
the SRS. In this stage, the system is decomposed into modules. The connection
between requirements and modules is listed in Table \ref{TblRT}.


\section{Module Decomposition} \label{SecMD}
Modules are decomposed according to the principle of ``information hiding'' while following
the MVC software design pattern. The \emph{Secrets} field in a module
decomposition is a brief statement of the design decision hidden by the
module while the \emph{Services} field specifies \emph{what} the module will do
without documenting \emph{how} to do it. For each module, a suggestion for the
implementing software is given under the \emph{Implemented By} title.
Only the leaf modules in described in the hierarchy will have to be implemented.
\subsection{Hardware Hiding Modules (\mref{mHH})}
\begin{description}
\item[Secrets:]The data structure and algorithm used to implement the virtual
  hardware.
\item[Services:]Serves as a virtual hardware used by the rest of the
  system. This module provides the interface between the hardware and the
  software so the system can use it to accept inputs and display outputs.
\item[Implemented By:] OS
\end{description}

\subsection{Behaviour-Hiding Module}
\subsubsection{Output Module (\mref{mOM})}
\begin{description}
\item[Secrets:]The format and structure of any output data.
\item[Services:] Converts the graphics created by the application into a viewable output to
the user.
\item[Implemented By:] PROJECT TETRIS
\end{description}

\subsubsection{Input Module (\mref{mIM})}
\begin{description}
\item[Secrets:] Handles the acception of input.
\item[Services:] Allows the reception of inputs from different sources (i.e. keyboard/mouse).
\item[Implemented By:] PROJECT TETRIS
\end{description}

\subsubsection{Game Graphics Module (\mref{mGG})}
\begin{description}
\item[Secrets:] Creates and converts the game into a graphical representation.
\item[Services:] Takes the logic of the game and transform it into graphical pieces that can be viewed
and understandable to the user.
\item[Implemented By:] PROJECT TETRIS
\end{description}

\subsubsection{Input Control Module (\mref{mIC})}
\begin{description}
\item[Secrets:]The format and structure of the input data.
\item[Services:] Converts the input data into the data structure used by the
  input parameters module.
\item[Implemented By:] PROJECT TETRIS
\end{description}

\subsection{Software Decision Module}
\subsubsection{Game Logic Module  (\mref{mGL})}
\begin{description}
\item[Secrets:] Algorithm that controls the behaviour of the game.
\item[Services:] Details rules on tetromino interaction as well as scoring. This can include
piece movement, collision, row clearing, and etc.
\item[Implemented By:] PROJECT TETRIS
\end{description}


\section{Traceability Matrix} \label{SecTM}
This section shows two traceability matrices: between the modules and the
requirements and between the modules and the anticipated changes.

\begin{table}[H]
\centering
\begin{tabular}{p{0.2\textwidth} p{0.6\textwidth}}
\toprule
\textbf{Req.} & \textbf{Modules}\\
\midrule
FR1 & \mref{mIM}, \mref{mGG}, \mref{mIC}, \mref{mGL}, \mref{mOM}\\
FR2 & \mref{mIM}, \mref{mGG}, \mref{mIC}, \mref{mGL}, \mref{mOM}\\
FR3 & \mref{mIM}, \mref{mGG}, \mref{mIC}, \mref{mGL}, \mref{mOM}\\
FR4 & \mref{mIM}, \mref{mGG}, \mref{mIC}, \mref{mGL}, \mref{mOM}\\
FR5 & \mref{mGG}, \mref{mGL}, \mref{mOM}\\
FR6 & \mref{mGL}\\
FR7 &  \mref{mGL}, \mref{mOM}\\
\bottomrule
\end{tabular}
\caption{Trace Between Requirements and Modules}
\label{TblRT}
\end{table}



\section{Use Hierarchy Between Modules} \label{SecUse}
\begin{table}[h!]
\centering
\begin{tabular}{p{0.3\textwidth} p{0.6\textwidth}}
\toprule
\textbf{A} & \textbf{Uses B}\\
\midrule
{M6} & {M2, M3, M4, M5}\\
\midrule
{M5} & {M2}\\
\midrule
{M6} & {M3}\\
\bottomrule
\end{tabular}
\caption{Module Hierarchy}
\label{TblMH}
\end{table}

\bibliographystyle{plainnat}
\bibliography{MG}

\end{document}