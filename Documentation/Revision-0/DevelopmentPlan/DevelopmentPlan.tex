\documentclass{article}
\usepackage{booktabs}
\usepackage{tabularx}
\title{SE 3XA3: Development Plan\\PROJECT TETRIS}
\author{Team 11
		\\ Daniel Agostinho agostd
		\\ Anthony Chang changa7
		\\ Divya Sridhar sridhad
}
\date{}
%% Comments

\usepackage{color}

\newif\ifcomments\commentstrue

\ifcomments
\newcommand{\authornote}[3]{\textcolor{#1}{[#3 ---#2]}}
\newcommand{\todo}[1]{\textcolor{red}{[TODO: #1]}}
\else
\newcommand{\authornote}[3]{}
\newcommand{\todo}[1]{}
\fi

\newcommand{\wss}[1]{\authornote{blue}{SS}{#1}}
\newcommand{\ds}[1]{\authornote{red}{DS}{#1}}
\newcommand{\mj}[1]{\authornote{red}{MSN}{#1}}
\newcommand{\cm}[1]{\authornote{red}{CM}{#1}}
\newcommand{\mh}[1]{\authornote{red}{MH}{#1}}

% team members should be added for each team, like the following
% all comments left by the TAs or the instructor should be addressed
% by a corresponding comment from the Team

\newcommand{\tm}[1]{\authornote{magenta}{Team}{#1}}

\begin{document}
\begin{table}[hp]
\caption{Revision History} \label{TblRevisionHistory}
\begin{tabularx}{\textwidth}{llX}
\toprule
\textbf{Date} & \textbf{Developer(s)} & \textbf{Change}\\
\midrule
10/18/2016 & Daniel Agostinho & Initial Draft\\
10/18/2016 & Anthony Chang & Initial Draft\\
10/18/2016 & Divya Sridhar & Initial Draft\\
\bottomrule
\end{tabularx}
\end{table}
\newpage
\maketitle
%\textcolor{red}{Always build your PDF file when committing your .tex file, there were changes in the code not in the PDF document. - CM} \\
\section{Team Meeting Plan}
%\textcolor{red}{Use a team name instead of Team 11 - CM} \\
%\textcolor{red}{Define method of tracking the weekly agenda - CM} \\
Since the start of this project, we have been meeting regularly on Monday afternoons from 5:30 to 7:00pm, at HG Thode Library. For each meeting there is an agenda, decided upon by all members of the group throughout the week. If a member comes up with an issue, it is discussed with the other group members about adding this problem to next week's agenda. The chair of the meetings is Daniel Agostinho. Meeting minutes are taken down by Anthony Chang and Divya Sridhar. At the end of the meeting, each member is assigned work with an appropriate deadline.
\section{Team Communication Plan}
%\textcolor{red}{ Use SMS instead of text as it is less ambiguous. - CM} \\
As a team, we have been using various forms of communication thus far, including SMS/phone, Facebook, and Git issues. Facebook is our primary source of collaboration and communication regarding the project deliverables, and Git issues is also used for specific communication regarding certain parts or sections of our project.
\section{Team Member Roles}
%\textcolor{red}{Include Anthony's Role, and specify roles pertaining to the project development - CM} \\
At this stage, all code and documentation work has been divided amongst all three group members. Project Tetris follows an MVC architecture, and each component (i.e. model, view, controller) has been distributed amongst the group. Daniel is the project lead, therefore he works on developing the code and improving the existing open-source project that is being used as reference. Divya and Anthony primarily work on documentation and completing the development plan, proof of concept, and so forth. These documents are also updated as the project progresses.
\section{Git Workflow Plan}
The Git workflow plan that will be used is the Feature-branch workflow. Because most of the work done by the members will be done independently throughout the week, the feature-branch workflow is optimal. Everyone can work on their own branch without causing large merge errors. Also before merging a pull request must be made which allows members to revise the commits. Labels will be used to identify different types of commits (ie. documentation, code, etc.) Milestones will be used to identify different tasks on Git. For example there will be different milestones for the Requirements Document, etc. \textcolor{red}{Who will manage the pull requests? When will they be done or how often? Provide more detail in this section. - CM} \\
\section{Proof of Concept Demonstration Plan}
\subsection{Will a part of the implementation be difficult?}
%\textcolor{red}{ Finish this section - CM} \\
At this stage of our project, the most difficult part of our implementation has been 
\subsection{Will testing be difficult?}
%\textcolor{red}{ Finish this section - CM} \\
As this program is a game, testing will be fairly difficult. Aside from using the testing frameworks such as JUnit and Mockito, we will overcome this testing difficulty by rigorous manual testing. We will incorporate feedback from other users who play and test our game, and thereby improve the quality of our game and gameplay for other users.
\section{Technology}
The programming language that will be used in this project is Java. Java was the programming language of choice because this is the language that all members were most comfortable with and the object oriented approach will help isolate future issues and challenges. \textcolor{red}{What issues? - CM} \\ The IDE to be used is Eclipse. Eclipse is an IDE that all members have been using for  years and have the most experience with. The testing framework to be used is JUnit. JUnit was a framework used in previous years and is again a comfortable option for team members to use. For document generation, JavaDoc will be used.
\textcolor{red}{Include details regarding libraries, frameworks ... etc.  - CM} \\
\section{Coding Style}
The coding style that Team 11 has decided to adopt is the Java style from Google. Link: https://google.github.io/styleguide/javaguide.html
\textcolor{red}{ Include rationale - CM} \\
\section{Project Schedule}
\textcolor{red}{Alter text to something more suitable for the reader, also actually provide a link - CM} \\
Provide a pointer to your Gantt Chart.
\section{Project Review}
\end{document}